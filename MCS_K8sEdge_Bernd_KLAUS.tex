%
% FH Technikum Wien
% !TEX encoding = UTF-8 Unicode
%

\documentclass[MSC,Master,english]{twbook}%\documentclass[Bachelor,BMR,ngerman]{twbook}
\usepackage[utf8]{inputenc}
\usepackage[T1]{fontenc}
% BERND: Additional Pkg
\usepackage{tcolorbox}
\usepackage{xcolor}
\usepackage{hyperref}
\usepackage{listings}

\lstset{
    captionpos=b,
    numberbychapter=false,
    caption=\lstname,
    frame=single,
    numbers=left,
    stepnumber=1,
    numbersep=2pt,
    xleftmargin=15pt,
    framexleftmargin=15pt, 
    %umberstyle=\tiny,
    tabsize=3,
    columns=fixed,
    basicstyle={\fontfamily{pcr}\selectfont\footnotesize},
    keywordstyle=\bfseries,
    commentstyle={\color[gray]{0.33}\itshape},
    stringstyle=\color[gray]{0.25},
    breaklines,
    breakatwhitespace
}
\lstloadlanguages{bash}

% Define Code-Color
\definecolor{codegreen}{rgb}{0,0.6,0}
\definecolor{codegray}{rgb}{0.5,0.5,0.5}
\definecolor{codepurple}{rgb}{0.58,0,0.82}
\definecolor{backcolour}{rgb}{0.95,0.95,0.92}

% Bitte in der folgenden Zeile den Zitierstandard festlegen
%\newcommand{\FHTWCitationType}{IEEE}
%\ifthenelse{\equal{\FHTWCitationType}{HARVARD}}{\usepackage{harvard}}{\usepackage{bibgem}}
%BERND: TBH that package is 20 years old, get some more up2date packages for cite
\usepackage[utf8]{inputenc}
\usepackage[english]{babel}
\usepackage{csquotes}
%\usepackage[backend=biber,style=ieee,bibstyle=ieee]{biblatex}
%\usepackage[backend=bibtex,style=numeric,bibstyle=ieee]{biblatex}
%ARCH Linux fix
\usepackage[backend=biber,style=numeric,sortcites,natbib=true,sorting=none]{biblatex}
\addbibresource{Literatur.bib}

%Formatieren des Quellcodeverzeichnisses
\makeatletter
% Setzen der Bezeichnungen für das Quellcodeverzeichnis/Abkürzungsverzeichnis
%in Abhängigkeit von der eingestellten Sprache
\providecommand\listacroname{}
\@ifclasswith{twbook}{english}
{%
    \renewcommand\lstlistingname{Code}
    \renewcommand\lstlistlistingname{List of Code}
    \renewcommand\listacroname{List of Abbreviations}
}{%
    \renewcommand\lstlistingname{Quellcode}
    \renewcommand\lstlistlistingname{Quellcodeverzeichnis}
    \renewcommand\listacroname{Abkürzungsverzeichnis}
}
%Wenn die Option listof=entryprefix gewählt wurde, Definition des Entyprefixes für das
%Quellcodeverzeichnis. Definition des Macros listoflolentryname analog zu listoflofentryname und
%listoflotentryname der KOMA-Klasse
\@ifclasswith{scrbook}{listof=entryprefix}
{% 
    \newcommand\listoflolentryname\lstlistingname
}{%
}
\makeatother
\newcommand{\listofcode}{\phantomsection\lstlistoflistings}


%
% Einträge für Deckblatt, Kurzfassung, etc.
%
\title{Kubernetes on the Edge}
\author{Bernd KLAUS, BA}
\studentnumber{2010303012}
\supervisor{Dipl.-Ing. Hubert Kraut}
\secondsupervisor{Dipl.-Ing. Andreas Happe}
\place{Wien}

\kurzfassung{
Kubernetes wird als Schweizer Armemesser der Container-Orchestrierung bezeichnet. Auch im Bereich Edge-Computing bietet der Dienst eine Vielzahl an unterschiedlichen Werkzeugen und Tools an, welche Teils unterschiedliche Strategien und Ansätze verfolgen. Die Auswahl reicht von einem zentralen Kubernetes-Cluster der verteilte Geräte, sogenannte „Leafs“, steuert bis hin zu vielen einzelnen und verteilten kleinen Clustern an der Edge, welche zentral gesteuert werden. Entscheidend ist es den richtigen Anwendungsfall zu erheben, um sich für die optimale Lösung entscheiden zu können. Ebenfalls spielen sicherheitstechnische Aspekte bei derart komplexen Umgebungen eine wichtige Rolle. Die vorliegende Arbeit gibt Einblicke und Entscheidungsgrundlagen sowie Empfehlungen hinsichtlich der IT-Security. Belegt werden die Angaben durch Implementierung eines Proof-of-Concepts
}
\schlagworte{Kubernetes, Edge-Computing, distributed System, Proof-of-Concept}


\outline{
Kubernetes is the de facto swiss-army-knfife for orchestrating conatiner-platforms. In addition, Kubernetes can also be faciliated for deploying devices as well as applications on top of it on the edge of the network. However, there are different methods for archiving comparable results. On the one hand a possible solution is to build a central instance managing small distributed and independent clusters, on the other hand a centralized cluster with just leafs on the edge may is a better fit. This results in the challenge to find the best solution for the desired environment respectively use-case. The following thesis is making use of "Design Science Research" to give introductions on how to choose the proper architecture for the aimed environment.
}
\keywords{Kubernetes, Edge-Computing, distributed System, Proof-of-Concept}

%
% Start the Thesis
%

\begin{document}
\maketitle
\chapter{Introcution}
\label{chap:introduction}
Because of \ac{IoT} Devices becoming more and more common, the number of devices capable of communicating with the \ac{WWW} increases rapidly. Consequently, also the overall traffic generated as well the amount of data which must be processed increases accordingly. Regarding this development Edge-Computing is the rising start trying to solve that issues. Thereby data is not processed centrally like in traditional datacenters, but it is tried to handle those data close to the user within several distributed systems. Because of this methodlogy only really necessary data is transmitted to a central instance for further treatment and those the processing-power as well as the bandwidth necessary for processing required data is reduced significant. \par It is expected that the number of IoT devices will continue to grow fast\cite{SotE21} over the coming years. Concomitant Edge-Computing also will become more important in the future and become an important role in modern \ac{IT} architectures. \par To be able to control distributed systems effectively \ac{K8S} is providing a lot of useful tools and functions. Fundamentally there are three different approaches regrading the architecture of how to build an Edge-Computing environment making use of \ac{K8S}:

\begin{itemize}
    \label{item:architecture}
    \item A centralized \ac{K8S} Cluster controlling many Leaf-Devices (Workers) on the Edge.
    \item Small and distributed \ac{K8S} Clusters running independent on the Edge controlled by a
    centralized Master-Instance.
    \item A Service-Mesh expanding the functionality of the \ac{K8S} networking stack.
\end{itemize}

\section{Problem area}
Problems arise when trying to find the proper architecture for a specific use-cases. There is no clear winner when comparing the above-mentioned different variants. Each of them  have their own pros and cons and may decide whether a project is successful or not. It is therefore all the more important to choose the proper architecture right before starting, changing the strategy in retrospect would take a lot of time and effort. However, there is no clear guidance on how to find the proper target environment, at least none which apply in general. Occasionally one finds recommendations for very specific use case, however the chance is slim low this findings fit your goals respectively enlighten the architecture decision. This leads us to the following research question.

\section{Research question}
\label{sec:rq}
This paper is going to answer the subsequent research questions:
\begin{enumerate}
    \item What are the main differences of the above mentiones architectures regarding   functionality, scalability, costs and security?
    \item Which decision criteria must be defined respectively examined to create a catalog capable
    of making choose the proper architecture easier for \ac{IT} managers as well as administrators?
    \item Is there a trend in which technology is most likely to be used?
\end{enumerate}

\section{Goal}
\label{sec:goal}
The main goal of this thesis is to highlight the pros and cons for each of the \hyperref[item:architecture]{architectures} defined in the \hyperref[chap:introduction]{Introduction}. The focus will mainly be on the geo-distribution aspect. Altough \ac{IoT} is playing a major role in pushing the development forward, however it is not considered further in the present work. To find the proper architecture, or at least recommandations what could fit best for different desired use-cases, a catalog will be defined. An important part will become the decision tree helping people making comprehensible decisions based on scientific research. The meain characteristics which are taken into account are scalability, state-of-the-art, handling, costs as well as security.

\section{Methodology}
\label{sec:methodology}
In the first part of the present work existing literature will be inspected. Related and relevant work will be examined accordingly and linked in the document. Also results will be incorporated to get out the most of it. In the second part a catalog with main criteria necessary for decision-making is defined. Part of this catalog will also be a decisicon-tree, mentioned in the previous chapter, to easily find the proper architecture. The last chapter deals with testing the defined criteria against real world examples making use of the \ac{DSR} methodology. The last chapter is getting the most focus because it is the area where new technics respectively architecture decision are finally verified and those proofs if the catalog is working as expected or not. In the latter case, the catalog will be revised to reflect the findings of the last step and re-examinated again using \ac{DSR}.




\chapter{State of the Art}
\label{chap:current}
The first chapter gives a brief introduction to the main technologies used respectively examinated in the later part of the document. If anyone is already familiar with the subject may you jump over to the \hyperref[sec:architecture]{Architecture} chapter.
\section{Technology}
\label{sec:technology}
\subsection{Kubernetes}
To promote mondern development and be able to implement continious deployment pipelines cumbersome monolytic applications are divied into many smaller units. Each of this units provides only one function. In order to establish the overall functionality, these units are communicate with eacher other and thus provide services or make us of other ones. This new method of delivering applications brings many advantages in terms of development but also introduce some new challenges and complexities regarding operation. To simplify the tasks around the management of this architecture, \ac{K8S} has established itself as the de facto standard \cite{k8ssurv}. Over the course of time, a broad community has developed around kubernetes and a number of additional tools and extensions have emerged as a result. The mopst promising solutions regrading geo-distribution respectively edge-computing are highlighted in the subsequent section \hyperref[sec:architecture]{Architecture}. In order to be able to interpret the results of the use-cases, as well as building the necessary basic understanding, the following functionalities and components of \ac{K8S} are of relevance.

\paragraph{Master Nodes} run the so called \textit{Control Plane} which is resposible for controlling the cluster itself and all the ressoruces within. The Control Plane consist of the following components \cite{k8scomp}.
\begin{itemize}
    \item \textit{kube-apiserver} acts as frontend web-interface responsible for controlling the \ac{K8S} cluster as well as the object inside the cluster. Tools like kubectl abstract the \textit{OpenAPI v2} endpoint and provide access in form of a simply understandable and usable \ac{CLI}. 
    \item \textit{etcd} represents a high-available and consistent key value store responsible for storing the actual state as well as the desired configuration of the cluster.
    \item \textit{kube-scheduler} is responsible for scheduling pods on the available worker nodes. Decision variables such as available ressources, affinity-rules and constraints are taken into account. However the default \textit{kube-scheduler} ist not aware of any latency between the worker nodes nor the pods communicating with each other. As discovered in the hereafter chapters, this appears to be an important variable for edge-deployments. However, some available white-papers already try to address those issues and showed possible solution by adopting a custom scheduler taking care of those values. More details on this can be found in the chapter \hyperref[chap:related]{Related Work}.
    \item \textit{kube-controller-manager} consists of a single compiled binary controling the status of nodes, jobs, service-accounts and endpoints as well as creating or removing them.
    \item \textit{cloud-controller-manager} represents the interface to the underlying cloud-platform. This allows kubernetes to create and/or configure load-balancers, routes and persistent-volumes on the underlying cloud-infrastructure. In a local environment such as e.g. minikube provide the \textit{cloud-controller-manager} becomes an optional component and is not required. The same may applies to edge-locations as those areas are outside the cloud most of the times.
\end{itemize}

\paragraph{Worker Nodes} manage the workload, i.e. run the actual application(s). These nodes are composed of the following, see list below, parts \cite{k8scomp}. It should be mentioned, that also the previously descripte \textit{Master Nodes} are executing those components because some of the core-componentes are containerized (pods) itself. 
\begin{itemize}
    \item \textit{kubelet} is an agent which assures that the container is executed properly inside their associated pods according to its specifications defined via \textit{PodSpec}. Also \textit{kubelet} is responsible for monitoring the healthy state of the containers.
    \item \textit{kube-proxy} uses the packet filters of the operating system underneath to forward traffic to the desired destination. The resulting access points, also called \textit{Services} in \ac{K8S}-jargon, can be made available either internally or externally.
    \item \textit{container runtime} is the part that finally executes the containers. The default runtime at time of writing is \textit{containerd}, however any runtime is supported that complies with the CRI specification \cite{cri-runtime}.
\end{itemize}

\paragraph{Kubernetes Objects} are persistent properties inside the \ac{K8S} ecosystem representing the state of your cluster. The most important feature of those objects is to describe the target environment in a declarative way. For this purpose, most of the time, YAML files are used. Kubernetes now ensures that the desired state of the environment is actually achieved and continuously monitors the required objects to meet those defined requirements. This mechanism is also ideal for distributed systems, such as edge computing, as availability can be checked at any time and a response can be made if necessary. Subsequent the main objects are cited starting with the smallest unit \cite{k8sconc}.
\begin{itemize}
    \item \textit{Containers} decouple the actual application and its dependencies from the underlying infrastructure. The main properties of those containers are there immutability and repeatability. This means that the container can be rebuilt at anytime resulting in an identical clone. Likewise, the code of a running container cannot be modified subsequently.
    \item \textit{Pods} include at least one or more \textit{Containers}. In the most scenarios a single pod consists of a single pod, in some cases a so-called sidecar container is used what make a pod consist of two containers. Containers which are in the same \textit{Pod} share the same local Socks as well as volumes mounted.
    \item \textit{Deployments}, \textit{Statefulsets} and \textit{Daemonsets} are responsible for ensuring the actual workload is provided, to achieve this they control and scale the assigned \textit{Pods}. When creating an application for \ac{K8S} it is most likely to create one of those objects, the \textit{Pods} and \textit{Containers} are merely an end product that is created from it.
\end{itemize}

\subsection{Edge-Computing}
latency
reliability and availability
low ressources
may bad network conditions

Target length: 1-2 S
\paragraph{Geo-Distribution}
short introduction - whats that!
Target length: 0,5-1 S




\section{Architecture}
\label{sec:architecture}
\subsection{Default}
Centralized Master and Workers at the Edge (Default!) K8s architecture
why is it called Default
challanges
Target length: 2-4 S (incl. picture)
\subsection{distributed K8s}
Cluster running indepented
Target length: 2-4 S (incl. picture)
\subsection{Service Mesh}
Using the Service-Mesh for Edge-Computing (Smar-Nic!)
Target length: 2-4 S (incl. picture)


\chapter{Design Science Research}
\label{chap:dsr}

\section{Methodlogy}
\label{sec:dsrmethode}
\subsection{Performed Tests}
Target length: 2 S

\section{Environment}
\label{sec:env}
describe the test-environemtn
Target length: 0,5-1 S

\section{Architecture}
\label{sec:dsrarchitecture}
\subsection{Default}
Target length: 3-4 S
\subsection{distributed K8s}
Target length: 3-4 S
\subsection{Service Mesh}
Target length: 3-4 S

\section{Use-Cases}
\subsection{Web-Application}
Target length: 1
\subsection{Enterprise VPN}
Target length: 1
\subsection{Distributed Database}
Target length: 1

\section{Analysis}
Target length: 4 S
\subsection{Relevant Magnitudes}
\subsection{Performed Tests}
\subsection{Outcome}
\subsection{Paraphrase}



\chapter{Catalog}
\label{chap:catalog}

\section{Decision Variables}
\label{sec:variables}
Target length: 1-2 S

\section{Decision Tree}
\label{sec:tree}
Target length: 1 S

\section{Exclusions and Special Cases}
\label{sec:exclusions}
Target length: 1 S



\chapter{Related Work}
\label{chap:related}
Target length: 3 S (all subsections)
\section{Kubernetes and the Edge?}
Some introdution to K8s at the Edge, highlighting the main Architectures.
\section{Extend Cloud to Edge with KubeEdge}
Descripes KubeEdge and its advantages
\section{Sharpening Kubernetes for the Edge}
Sharpening Kubernetes for the Edge
Make Kubernetes aware of the latency between the nodes at the edge.
\section{Ultra-Reliable and Low-Latency Computing in the Edge with Kubernetes}
Similar to the paper before. Latecny awar pod deploymentm, but you also can deploy to regions and an custom re-scheduler is implementated taking care of redeploying when one node fails.
Clustering node-groups based on latency.


\chapter{Results}
\label{chap:results}

Target length: 3 S (all together)
\section{Findings}
\label{sec:findings}

\section{Conclusio}
\label{sec:conclusio}

\section{Discussion and further research}
\label{sec:discuss}

\paragraph{Notes}
---to-be-removed---
Sites:
- longest: 50 (may i need even more)
- shortest: 36 (zu wenig)
-------------------


%
% Hier beginnen die Verzeichnisse.
%
\clearpage
%\ifthenelse{\equal{\FHTWCitationType}{HARVARD}}{}{\bibliographystyle{biblatex}}
%\bibliography{Literatur}
\printbibliography 
\clearpage

% Das Abbildungsverzeichnis
\listoffigures
\clearpage

% Das Tabellenverzeichnis
\listoftables
\clearpage

% Das Quellcodeverzeichnis
\listofcode
\clearpage

\phantomsection
\addcontentsline{toc}{chapter}{\listacroname}
\chapter*{\listacroname}
\begin{acronym}[XXXXX]
    \acro{IT}[IT]{Information Technology}
    \acro{WWW}[WWW]{world wide web}
    \acro{K8S}[K8S]{Kubernetes}
    \acro{IoT}[IoT]{Internet-of-Things}
    \acro{DSR}[DSR]{Design Science Research}
    \acro{CLI}[CLI]{Command Line Interface}
\end{acronym}

%
% Hier beginnt der Anhang.
%
\clearpage
\appendix
\chapter{Anhang A}
\clearpage
\chapter{Anhang B}
\end{document}