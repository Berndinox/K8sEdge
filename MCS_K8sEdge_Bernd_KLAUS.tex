%
% FH Technikum Wien
% !TEX encoding = UTF-8 Unicode
%

\documentclass[MSC,Master,english]{twbook}%\documentclass[Bachelor,BMR,ngerman]{twbook}
\usepackage[utf8]{inputenc}
\usepackage[T1]{fontenc}
% BERND: Additional Pkg
\usepackage{tcolorbox}
\usepackage{xcolor}
\usepackage{hyperref}
\usepackage{listings}


\lstset{
    captionpos=b,
    numberbychapter=false,
    caption=\lstname,
    frame=single,
    numbers=left,
    stepnumber=1,
    numbersep=2pt,
    xleftmargin=15pt,
    framexleftmargin=15pt, 
    %umberstyle=\tiny,
    tabsize=3,
    columns=fixed,
    basicstyle={\fontfamily{pcr}\selectfont\footnotesize},
    keywordstyle=\bfseries,
    commentstyle={\color[gray]{0.33}\itshape},
    stringstyle=\color[gray]{0.25},
    breaklines,
    breakatwhitespace
}
\lstloadlanguages{bash}

% Define Code-Color
\definecolor{codegreen}{rgb}{0,0.6,0}
\definecolor{codegray}{rgb}{0.5,0.5,0.5}
\definecolor{codepurple}{rgb}{0.58,0,0.82}
\definecolor{backcolour}{rgb}{0.95,0.95,0.92}

% Bitte in der folgenden Zeile den Zitierstandard festlegen
\newcommand{\FHTWCitationType}{IEEE}
\ifthenelse{\equal{\FHTWCitationType}{HARVARD}}{\usepackage{harvard}}{\usepackage{bibgerm}}


%Formatieren des Quellcodeverzeichnisses
\makeatletter
% Setzen der Bezeichnungen für das Quellcodeverzeichnis/Abkürzungsverzeichnis
%in Abhängigkeit von der eingestellten Sprache
\providecommand\listacroname{}
\@ifclasswith{twbook}{english}
{%
    \renewcommand\lstlistingname{Code}
    \renewcommand\lstlistlistingname{List of Code}
    \renewcommand\listacroname{List of Abbreviations}
}{%
    \renewcommand\lstlistingname{Quellcode}
    \renewcommand\lstlistlistingname{Quellcodeverzeichnis}
    \renewcommand\listacroname{Abkürzungsverzeichnis}
}
% Wenn die Option listof=entryprefix gewählt wurde, Definition des Entyprefixes für das
%Quellcodeverzeichnis. Definition des Macros listoflolentryname analog zu listoflofentryname und
%listoflotentryname der KOMA-Klasse
\@ifclasswith{scrbook}{listof=entryprefix}
{%
    \newcommand\listoflolentryname\lstlistingname
}{%
}
\makeatother
\newcommand{\listofcode}{\phantomsection\lstlistoflistings}


%
% Einträge für Deckblatt, Kurzfassung, etc.
%
\title{Kubernetes on the Edge}
\author{Bernd KLAUS, BA}
\studentnumber{2010303012}
\supervisor{Dipl.-Ing. Hubert Kraut}
\secondsupervisor{Dipl.-Ing. Andreas Happe}
\place{Wien}

\kurzfassung{
Kubernetes wird als Schweizer Armemesser der Container-Orchestrierung bezeichnet. Auch im Bereich
Edge-Computing bietet der Dienst eine Vielzahl an unterschiedlichen Werkzeugen und Tools an, welche
Teils unterschiedliche Strategien und Ansätze verfolgen. Die Auswahl reicht von einem zentralen 
Kubernetes-Cluster der verteilte Geräte, sogenannte „Leafs“, steuert bis hin zu vielen einzelnen und
verteilten kleinen Clustern an der Edge, welche zentral gesteuert werden. Entscheidend ist es den 
richtigen Anwendungsfall zu erheben, um sich für die optimale Lösung entscheiden zu können.
Ebenfalls spielen sicherheitstechnische Aspekte bei derart komplexen Umgebungen eine wichtige Rolle.
Die vorliegende Arbeit gibt Einblicke und Entscheidungsgrundlagen sowie Empfehlungen hinsichtlich
der IT-Security. Belegt werden die Angaben durch Implementierung eines Proof-of-Concepts
}
\schlagworte{Kubernetes, Edge-Computing, distributed System, Proof-of-Concept}

\outline{
Kubernetes is the de facto swiss-army-knfife for orchestrating conatiner-platforms. In addition,
kubernetes is also offering a toolset necessary for deploying on the edge. However, there are
different methods for archiving comparable results. On the one hand a possible solution is to build
a central instance managing small distributed and independent cluster, on the other hand a
centralized cluster with just leafs on the edge may is a better fit. The challenge is to dine the
best solution for the desired environment. The following thesis is making use of "Design Science
Research" to give introductions on how to choose the proper environment.
}
\keywords{Kubernetes, Edge-Computing, distributed System, Proof-of-Concept}

%
% Start the Thesis
%

\begin{document}

\maketitle

\chapter{Introcution}
Because of IoT Devices becoming more and more common, the number of devices capable of communicating
with the \ac{WWW} increases rapidly. Consequently, also the overall traffic generated as well the
amount of data which must be processed increases accordingly. Regarding this development
Edge-Computing is the rising start trying to solve that issues. Thereby data is not processed
centrally like in traditional datacenters, but it is tried to handle those data close to the user 
within several distributed systems. Because of this method only really necessary data is transmitted
to a central instance for further treatment and those the processing-power as well as the bandwidth
necessary for processing required data is reduced significant. \par
It is expected that the number of IoT devices will continue to grow fast\cite{SotE21} over the
coming years. Concomitant Edge-Computing also will become more important in the future and play
an important role in modern \ac{IT} architectures. \par
To be able to control distributed systems effectively \ac{K8S} is providing a lot of useful tools
and functions. Fundamentally there are two different approaches regrading the architecture of how To
build an Edge-Computing environment using \ac{K8S}:

\begin{itemize}
    \item One centralized \ac{K8S} Cluster controlling many Leaf-Devices on the Edge.
    \item Small and distributed \ac{K8S} Clusters running independent on the Edge controlled by a
    centralized Master-Instance.
\end{itemize}

\section{Problem area}
Problems arise when trying to find the proper architecture for a specific use-cases. There is no
clear winner when comparing the above-mentioned different variants. Each of the both architectures
have their own pros and cons and may decide whether a project is successful or not. It is therefore
all the more important to choose the proper architecture right before starting, changing the
strategy in retrospect would take a lot of time and effort. However, there is no clear guidance on
how to find the proper target environment, at least none which apply in general. Occasionally one
finds recommendations for very specific use case, however the chance is slim low this findings fit
your goals respectively enlighten the architecture decision. This leads us to the following research
question.

\section{Research question}
This paper is going to answer the subsequent research questions:
\begin{enumerate}
    \item What are the main differences between the both defined main variants regarding
    functionality, scalability, costs and security?
    \item Which decision criteria must be defined respectively examined to create a catalog capable
    of making choose the proper architecture easier for \ac{IT} managers as well as administrators.
    \item Can two Proof-of-Concept environments, compared using the Design Science Research pattern,
    prove the accuracy of defined criteria accordingly.
\end{enumerate}

\section{Methodology}
In the first part of the present work existing literature will be inspected. Related and relevant
work will be examined accordingly and linked in the document. Also results will be incorporated to
get out the most of it. In the second part a catalog with main criteria necessary for
decision-making is defined. The last chapter is testing those criteria against real world examples
making use of the Design Science Research methodology. The last chapter is getting the most focus
because it is the area where new technics respectively architecture decision are finally verified
and those proof if the catalog is working as expected or not.

%
% Hier beginnen die Verzeichnisse.
%
\clearpage
\ifthenelse{\equal{\FHTWCitationType}{HARVARD}}{}{\bibliographystyle{gerabbrv}}
\bibliography{Literatur}
\clearpage

% Das Abbildungsverzeichnis
\listoffigures
\clearpage

% Das Tabellenverzeichnis
\listoftables
\clearpage

% Das Quellcodeverzeichnis
\listofcode
\clearpage

\phantomsection
\addcontentsline{toc}{chapter}{\listacroname}
\chapter*{\listacroname}
\begin{acronym}[XXXXX]
    \acro{IT}[IT]{Information Technology}
    \acro{WWW}[WWW]{world wide web}
    \acro{K8S}[K8S]{Kubernetes}
\end{acronym}

%
% Hier beginnt der Anhang.
%
\clearpage
\appendix
\chapter{Anhang A}
\clearpage
\chapter{Anhang B}
\end{document}